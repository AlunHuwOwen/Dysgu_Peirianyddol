\chapter{Cyflwyniad}\label{cha:introduction}

Mi fyddwn yn ysgrifennu fy mhrosiect trydydd flwyddyn am algorithmau dysgu peirianyddol ag sut i'w defnyddio yn R ac Python. Fydd y prosiect yn cael ei ysgrifennu trwy cyfrwn y Gymraeg. Mi fyddwn yn creu traethawd a gwefan i gyfathrebu gwybodaeth am yr algorithmau. Dwi am ysgrifennu am glystyru $k$-cymedr, atchweliad logistaidd ag dosbarthiad na\"{i}f Bayes.

Bydd y tiwtorialau yn cael ei ddangos drwy'r ieithoedd rhaglennu Python ag R, gan taw nhw yw'r ddwy iaith fwyaf poblogaidd ar gyfer dysgu peirianyddol a gwyddor data yn \^{o}l arolwg Kaggle nol yn 2018 \cite{kagglesurvey}. Mae hefyd ganddynt trwyddedau am ddim, sy'n bwysig ar gyfer hygrychedd. Rheswm arall i ddysgu defnyddio'r ieithoedd rhaglennu yma yw bod nhw'n god agored sy'n wahanol i SQL er enghraifft. Gan ei fod nhw'n god agored poblogaidd, mae'n meddwl fod nhw'n cael ei diweddaru yn aml. Cafodd R ei greu gan ystadegwyr a gan fy mod yn creu tiwtorialau am ddysgu peirianyddol o'r ochr ystadegaeth, mae'n gymhelliad i ddefnyddio R. Ar gyfer dysgu Python, nol yn 2019 roedd Python yr iaith rhaglennu ail fwyaf i gael ``pull requests'' ar Github \cite{canran-github}, a Github oedd y gwesteiwr fwyaf o god ffynhonnell yn y byd. Mae Python ag R yn cael ei ystyried i fod yn hawdd dysgu a deall yn gymhariaeth gydag ieithoedd rhaglennu eraill. Gwelwn yn y tiwtorialau i ddod nad yw gweithredu'r algorithmau yn yr ieithoedd hyn yn gymhleth.

\section{Beth yw Dysgu Peirianyddol}\label{sec:intro_pd}

Yn syml, dysgu peirianyddol yw algorithmau i optimeiddio rhyw feini prawf gan ddefnyddio data. Mae rhain yn cynnwys model wedi'i ddiffinio o rhai paramedrau mesuradwy, a'r darn dysgu fydd i optimeiddio gyda pharch tuag at y paramedrau hyn. Gall y model fod yn un disgrifiadol o'r data, neu un sy'n rhagfynegi rhyw agwedd o'r data. Mae dysgu peirianyddol yn defnyddio ystadegaeth i adeiladu'r modelau mathemategol, ac mae cyfrifiadureg yn edrych mwy i mewn i effeithiolrwydd y proses.\cite{dysgu-peirianyddol2}

Nid yw dysgu peirianyddol yn faes newydd, mae wedi bod o gwmpas ers y 50au pan wnaeth Arthur Samuel o IBM creu rhaglen ar y cyfrifiadur i chwarae'r gem draffts. Yr amser hwn cafodd y term ei bathu. Yna drwy ddatblygiadau technolegol diweddar, mae posibilrwyddau ddysgu peirianyddol bron yn ddiddiwedd. Ers cael ei sefydlu yn y 50au, cymerodd tan 1997 i ddatblygu rhaglen a all guro'r chwaraewr gwyddbwyll orau yn y byd. Nid yn unig yw dysgu peirianyddol yn cael ei ddefnyddio i greu rhaglenni gemau, mae nawr yn cael ei ddefnyddio ym mhob math o raglenni megis llawer o apiau ar eich ff\^{o}n fel ``Google Maps'', ``Uber'' i ``Netflix'' \cite{cymwysiadaudysgupeirianyddol}.

Mae sylfaeni ddysgu peirianyddol yn cael eu defnyddio yn prosesu iaith naturiol a dadansoddi sentiment\cite{technolegau-iaith}. Mae'r proseses yma yn cael ei ddefnyddio i wneud ffwythiannau megis adnabod lleferydd, creu tecst i leferydd a chyfieithiad peirianyddol.
Yn ogystal mae dysgu peirianyddol yn cael ei ddefnyddio i brosesu lluniau, mae'r defnydd yma'n cael ei weld yn aml gyda systemau adnabod wynebau.

Mae dysgu peirianyddol yn cael ei ddefnyddio yn aml i ddadansoddi data pryd bynnag gennym ddata mawr. Mae ein llywodraeth yn defnyddio dysgu peirianyddol yn aml, mae adran actiwari y llywodraeth yn defnyddio dysgu peirianyddol i ddatblygu mewnwelediadau i'w problemau. Hyd yn hyn maen nhw wedi defnyddio dysgu peirianyddol i rannu cynlluniau pensiwn i grwpiau gyda phriodweddau tebyg ac wedi rhagfynegi cyflog graddedigion yn y dyfodol. \cite{GAD}  

Mae deallusrwydd artiffisial wedi tyfu yn esbonyddol yn ddiweddar gyda dysgu peirianyddol. Mae deallusrwydd artiffisial yn y newyddion drwy'r adeg oherwydd datblygiadau parhaus. Yn ddiweddar rydym wedi gweld moduron heb yrrwr, diagnosau meddygol, a `chat bots'.

Categoreiddiwn algorithmau dysgu peirianyddol i ddwy brif fath - dysgu o dan orchwyliaeth, a dysgu heb orchwyliaeth. Maent yn wahanol yn eu pwrpas a'u dulliau.

\subsection{Dysgu dan Oruchwyliaeth}

Diffiniwn $\alpha$ i fod y set o bob label, gall y labeli fod yn arwahanol neu yn ddi-dor, yna diffiniwn $\beta$ fod y fector o ddimensiwn $D \in \mathbb{Z}_{+}$. Gadewch i'n data fod nifer o barau o bwyntiau $(\mathbf{x}_{i},y_{i})$. Yn y fan hyn mae $y_{i} \in \alpha$. Fydd $\mathbf{x}_{i} \in  \beta$ yn fector gyda gwerthoedd ar gyfer priodoleddau gwahanol. Y nod ar gyfer dysgu dan oruchwyliaeth yw dysgu'r mapiad o $\mathbf{x}$ i $y$.\cite{dysgu-peirianyddol}

Gwnawn hyn trwy hollti'r data i mewn i set hyfforddi a set profi, lle mae'n bwysig i dybio pob pwynt wedi'i samplu'n annibynnol a'i dosbarthu o ddosraniad dros $\alpha \times \beta$. Defnyddiwn y set hyfforddi i rhedeg yr algorithm dysgu peiranyddol a chanfod y mapiad. Defnyddiwn y set profi er mwyn gwirio'r mapiad hyn. Fel arfer defnyddiwn tua 70\% o'r data fel y set hyfforddi, a'r 30\% gweddill fel y set profi, er taw tra-baramedr (hyper-parameter) yr hon ac felly gellid ei ddewis.

Mae'r ffurfiau wahanol o ddysgu dan oruchwyliaeth yn cael ei rhannu i ddau faes yn \^{o}l sut fath o ddata sydd gennym. Os yw'r label yn data arwahanol, mae gennym ddosbarthiad (classification). Os yw'r label yn data di-dor, mae gennym atchweliad (regression). Mae yna lwyth o wahanol fathau o ddosbarthiadau ag atchweliadau; dyma ambell o enghreifftiau ohonynt:

\begin{multicols}{2}
\textbf{Dosbarthiad:}

\begin{itemize}
	\item Coed penderfyniadau
	\item Dosbarthiad na\"{i}f Bayes 
	\item $K$ cymydog agosaf
\end{itemize} 

\textbf{Atchweliad:}

\begin{itemize}
	\item Atchweliad logistaidd
	\item Atchweliad llinol
	\item Atchweliad Poisson
\end{itemize}
\end{multicols}

\subsection{Dysgu heb Oruchwyliaeth}

Diffiniwn $\gamma$ i fod yn ddosraniad. Gadewch i'n data fod $X = (x_{1}, \dots, x_{n})$ sy'n dynodi $n$ pwyntiau lle mae $x_{i} \in \gamma$ ag $i \in \{ 1, \dots, n \}$. Tybiwn fod yr enghreifftiau $x_{i}$ wedi'i samplu'n annibynnol a'i dosbarthu o ddosraniad unfath ar $\gamma$. Y nod o ddysgu heb oruchwyliaeth yw amcangyfrif dwysedd o'r dosraniad sy'n debygol o fod wedi creu $X$.\cite{dysgu-peirianyddol}

Mae'r fatha boblogaidd o ddysgu heb oruchwyliaeth yn cymryd ffurf wannach o'r syniad hyn. Dyma ambell i enghraifft ar ffurfiau gwahanol o ddysgu heb oruchwyliaeth:

\begin{itemize}
	\item Clystyru
	\item Lleihad dimensiwn
	\item Model Markov cudd
\end{itemize}

\subsection{Dysgu atgyfnerthol}

Mae dysgu atgyfnerthol yn dysgu drwy wrando ar adborth o'r system ei hyn. Fydd y peiriant yn dysgu i wneud dulliau sy'n arwain tuag at adborth da ac yna yn osgoi unrhyw dullai caiff adborth gwael. Mae hyn yn wahanol i ddysgu dan oruchwyliaeth gan fod does ddim gymaint o ddibyniaeth ar y ddata hyfforddi. Mae'n wahanol i ddysgu heb oruchwyliaeth gan ein bod angen dewis pryd i dderbyn adborth o'r system \cite{technolegau-iaith}. 

Enghraifft o ddysgu atgyfnerthol yw prosesau penderfynu Markov, mae'n cael ei ddiffinio fel proses stocastig dan reolaeth. Mae prosesau penderfynu Markov yn cael ei ddiffinio gan y plyg $(S,A,T,p,r)$. Yn y plyg mae'r newidynnau yn cael ei ddiffinio fel: $S$ yw'r gofod sy'n cynrychioli esblygiad y prosesau, $A$ yw'r set o bob gweithred sy'n bosib, $T$ yw'r set o amseroedd rhwng dewisiadau, $p$ sy'n dynodi'r ffwythiant tebygolrwydd ar gyfer y trosglwyddiad cyflwr ac $r$ sy'n dynodi'r ffwythiant wobrwyo i'r trosglwyddiad cyflwr. Mae prosesau penderfynu Markov yn efelychu system ac yna yn dewis gweithrediadau ar hap ac yn penderfynu ansawdd y dewis drwy'r ffwythiant wobrwyo. Felly fydd y system yn optimeiddio i ddewis yr opsiynau fydd yn allbwn y wobr orau ac felly yn dewis yr opsiwn o'r ansawdd orau.\cite{PPM} 

\section{Adnoddau Cyfrwng Cymraeg ar gyfer Dysgu Peirianyddol}

Dyma'r adnoddau Cymraeg sydd ar gael yn barod yn y faysydd Deallusrwydd Artiffisial, Cyfrifiadureg ag Mathemateg:

\subsection{Gwerthuso adnoddau sy'n bodoli}
\begin{table}[htbt!]

\begin{center}
\rotatebox{90}{%
\begin{tabular}{ | p{0.24\textwidth} | p{0.24\textwidth} | p{0.24\textwidth} | p{0.24\textwidth} | p{0.05\textwidth} | }
\hline
\textit{\textbf{Adnodd}} & \textit{\textbf{Awdur}} & \textit{\textbf{Fformat}} & \textit{\textbf{Cynulleidfa Darged}} & \textit{\textbf{Linc}}\\
\hline
Cwrs `Cyfrifiadureg ar gyfer Mathemateg' yn Python & Dr Vince Knight a Dr Geraint Palmer & Tiwtorial ar wefan & Dechreuwyr mewn codio Python & \cite{Cyfrifiadureg-maths}\\
\hline
Cyrsiau allgyrsiol yn Scratch, HTML, CSS ag Python & Code Club & Tiwtorialau ar y w\^{e} & Plant rhwng $9$ ac $13$ & \cite{codeclub} \\
\hline
Adnoddau ar lawer o testynau wahanol yn cyfrifiadureg & Technocamps & Mewn ffurdd pdf & Plant yn ysgol/coleg & \cite{technocamps} \\
\hline
Cwrs mewn sgiliau ymchwil cyfrifiadurol & Dr Geraint Palmer & Tiwtorialau ar y w\^{e} & Ymchwilwyr & \cite{python-sgiliauymchwil} \\
\hline
Gwybodath ar gydrannau pwysicaf technolegau iaith & Uned technoleg iaith Prifysgol Bangor a Cymen Cyf. & Llawlyfr ar y w\^{e}  & Myfyrwyr, datblygwyr neu academyddion heb gefndir yn y maes & \cite{technolegau-iaith} \\
\hline
ap Botio i hybu rhaglennu i plant & Tinopolis & ap ar cynnyrch apple & Plant rhwng $7$ ac $11$ & \cite{botio} \\
\hline
Enghreifftiau a thiwtorialau o cynnyrch technolegau iaith  &  Uned technoleg iaith Prifysgol Bangor a Cymen Cyf. & Ystorfa ar github & Myfyrwyr, datblygwyr neu academyddion heb gefndir yn y maes & \cite{github-technolegauiaith} \\
\hline
Tiwtorialau Python & Dr Geraint Palmer a Stephanie Jones & Fideos ar Youtube & Dechreuwyr yn rhaglennu gyda Python & \cite{youtube} \\
\hline
Tiwtorialau SQL & pl/sql tiwtorial & Wefan gyda tiwtorialau & Dechreuwyr yn SQL & \cite{sql} \\
\hline
Nodiadau agored am algebra llinol & Dr Alun Morris & Nodiadau (pdf) & Myfyrwyr prifysgol & \cite{Algebra-llinol}\\
\hline
Amrhyw o adnoddau Mathemategol yn dilyn cwricwla CBAC & Dr Gareth Evans & Wefan gyda linciau i wahanol fathau o adnoddau & Myfyrwyr rhwng $11$ a $18$ & \cite{mathemateg} \\
\hline
\end{tabular}%
}
\end{center}\label{fig:tabl}
\caption{Tabl o adnoddau sydd allan yn barod.}
\end{table}


Mae Tabl~\ref{fig:tabl} wedi'i rhannu i adnoddau codio Cymraeg ag i adnoddau mathemategol Cymraeg. Fel gwelwn yn ebrwydd, mae'r nifer o adnoddau codio yn llawer fwy nag rheina o adnoddau mathemategol. Mae'r adnodd \cite{Cyfrifiadureg-maths} yn cychwyn da i unrhyw un sydd gyda diddordeb o gychwyn codio yn Python, mae'n defnydd gwych ar gyfer myfyrwyr mathemateg israddedig gan ei fod yn benodol i fathemateg. Yn ogystal i'r cwrs yma mae gennym hefyd bach o gyflwyniad i Python ar y wefan Sgiliau Ymchwil Ailhynhyrchiadwy \cite{python-sgiliauymchwil}, er bod hyn yn canolbwyntio mwy ar sgiliau fwy eang megis rheolaeth fersiwn a datblygu meddalwedd ymchwil.

\clearpage

Gwelwn yn ogystal fod yna adnoddau ar gyfer rhaglennu yn Python yn Gymraeg ar gael ar YouTube. Ar sianel Geraint Palmer \cite{youtube} mae yna gasgliad o diwtorialau Python. Mae'r tiwtorialau yn cychwyn gyda'r sylfaen o raglennu yn Python ag yn gorffen gyda mynd dros yr algorithm genetig, sy'n mynd law yn llaw gyda fy mhrosiect.
Mae gan y wefan codeclub amrywiaeth eang o adnoddau ar gyfer codio \cite{codeclub}. Mae yna gyrsiau ar HTML, CSS, Python a Scratch. Yn ogystal mae yna brosiectau pellach sy'n gweithredu Raspberry Pi. 

Gwelwn gydag adnodd technocamps \cite{technocamps}, fod yna amrywiaeth eang o adnoddau yn fan hyn fyd. Mae'r wefan wedi'i thargedu i oedran h\^{y}n na'r wefan codeclub. Mae gan technocamps cyrsiau cychwynnol fel y canlynol: CS 101, Deallusrwydd Artiffisial, Greenfoot (Java), Python a Scratch.

Ar wefan technolegau iaith \cite{technolegau-iaith}, mae yna gyflwyniad i'r darnau fwyaf sylfaenol i dechnolegau iaith. Mae yna lawlyfr yn cynnwys gwybodaeth am ddeallusrwydd artiffisial, dysgu dwfn a phrosesu iaith naturiol, mae'n rhoi cyflwyniad i fewn i'r ochr damcaniaethol ohono. Yn rhedeg yn gyfagos i'r wefan yma yw'r ystorfeydd ar GitHub \cite{github-technolegauiaith}. Yn yr ystorfeydd mae yna diwtorial ar sut i greu robot sgwrsio syml drwy ``turing test lessons'', mae'n addas i rhai yng nghyfnod allweddol $2$ a $3$. Mae'n cynnwys tair gwers, un ar sut mae cyfrifiaduron yn meddwl, yr ail ar ydy cyfrifiaduron yn gallu meddwl drostynt eu hynain ag y dwythaf am creu robot sy'n sgwrsio drwy'r defnydd o Raspberry Pi.

Yn y siop ap Apple, fedrem lawrlwytho'r ap botio \cite{botio}. Mae'n ap am chwilota'r gofod am blanedau newydd drwy raglennu. Mae'r ap wedi'i ariannu gan Lywodraeth Cymru ag ar gael am ddim. Cafodd ei greu ar gyfer plant sydd \^{a} diddordeb cael i mewn i raglennu.

Ar yr ochr mathemategol, mae'r adnodd am ddim gan Alun Morris \cite{Algebra-llinol} yn trafod y sylfeini eu hangen i astudio algebra llinol yn mhrifysgol. Mae'r adnodd yma yn enghraifft wych o'r fath o adnoddau mae'r iaith Gymraeg angen fwy ohono. Mae prinder iawn ar adnoddau o'r ansawdd yma yn agored i'r cyhoedd. Mae yna tair prifysgol yn cynnig cyrsiau mathemateg drwy'r Coleg Cymraeg ond dydi dim un yn darparu eu nodiadau drwy borth Coleg Cymraeg. 
Serch hynny mae yna gyfoeth o adnoddau mathemateg Cymraeg ar gyfer addysg o dan $18$ oed, gwelwn hyn gyda'r nifer mawr o adnoddau ar wefan Dr Gareth Evans o Ysgol Creuddyn \cite{mathemateg}.  

Gwelwn fod yna llawer o adnoddau ar gyfer y maes mathemateg a chyfrifiadureg ar gyfer addysg ysgol. Mae yna ddigon o adnoddau i alluogi cenhedlaeth newydd i astudio'r sylfaen gofynnol drwy'r ysgol i astudio unrhyw un o'r ddau yn brifysgol. Unwaith fyddem yn cyrraedd lefel addysg prifysgol, mae'r prinder yn amlwg. Gan fod testunau dysgu peirianyddol yn cynnwys mathemateg o lefel uwch, mae rhaid cael sylfaen datblygedig o destunau mathemateg i'w ddeall. Gallwn ddweud yr un peth am yr angenrheidrwydd o sylfaen datblygedig o gyfrifiadureg. Y broblem sydd gennym yw does yna ddim yr adnoddau yw gwneud gyda mathemateg lefel uwch nag rhaglennu yn R yn y Gymraeg.

\subsection{Yr angen am adnoddau Cyfrwng Cymraeg}

I gychwyn, rwyf eisiau creu gwefan sy'n cynnwys tiwtorialau Cymraeg am algorithmau dysgu peirianyddol oherwydd y prinder ohonyn. Mae yna gymaint o adnoddau ar gyfer pob math o algorithm trwy gyfrwng Saesneg, ond yn Gymraeg, does yna ddim byd o'r fath! Fel gwelwn yn Nhabl~\ref{fig:tabl}, does yna ddim llawer o adnoddau Cymraeg sy'n cynnwys deunydd datblygedig ar gael i'r cyhoedd. O fy mhrofiad personol i, rwyf yn gwybod fod prifysgolion yng Nghymru gydag adnoddau drwy'r cyfrwng Cymraeg ar gael i'w myfyrwyr nhw yn unig. Teimlaf fod hyn yn atal y parhad o addysg ar \^{o}l i fyfyrwyr gorffen eu hastudiaethau ffurfiol oherwydd nad oes adnoddau am ddim ar gael. 

Ers i'r iaith Gymraeg cael ei rhoi ar sylfaen gyfartal i Saesneg yn Gymru yn 1993 a chael gwasanaethau cyhoeddus yn y ddwy iaith, mae wedi bod effaith domino i'r sector preifat. Mae prifysgolion yn gyrff preifat ac felly nid yw'n angenrheidiol iddyn nhw ddilyn \cite{prifysgolion}, ond yn 2011 cafodd y coleg Cymraeg ei sefydlu i ornest a hyn. Mae'r coleg yn hybu prifysgolion Cymru i newid i fod yn ddwyieithog drwy weithio yn bartneriaeth gyda nhw. Hyd at heddiw mae yna fwy o fodiwlau yn cael ei gyfieithu a fwy o fodiwlau yn cael ei chynnig drwy gyfrwng y Gymraeg. Ar gyfer y garfan fydd yn graddio yn 2020 dim ond naw modiwl oedd ar gael drwy'r cyfrwng Cymraeg a rhan fwyaf ohonyn nhw yn y flwyddyn gyntaf.

Rheswm arall i wneud yn Gymraeg yw cefnogi prosiect y llywodraeth a'r wlad i gael miliwn o siaradwyr Cymraeg erbyn 2050 \cite{cymraeg2050}. I gymhorthi'r prosiect yma cafodd cynllun ei rhoi allan yn hybu'r defnydd a chynnyrch o adnoddau technolegol Cymraeg \cite{cymraeg2050tech}. Yn ogystal i greu adnoddau mae'r cynllun yn targedu y datblygiad o ddeallusrwydd artiffisial drwy edrych ar brosesu iaith naturiol yn bennaf. Mae creu'r adnodd yma yn Gymraeg yn bodloni'r ddwy adran o'r cynllun, gobeithio bydd yn hybu fwy i gychwyn creu adnoddau addysg uwch yn y maes yma. Un ateb cyflym i'r prinder fysa os fysa'r prifysgolion yn gadael ei adnoddau allan am ddim ar borth coleg Cymraeg, fel bod rhai prifysgolion yn yr Unol Daleithiau America lle mae yna ystorfa o gyrsiau gydag adnoddau agored ar wefannau fel Coursera \cite{coursera}.

\section{Strwythyr}

Fydd y traethawd yma yn cynnwys tair pennod ar algorithm dysgu peirianyddol. Yn pob pennod fydd yna gefndir ar beth yw'r algorithm a pryd fyddem yn ei ddefnyddio, yn ogystal fydd yna gefndir ar sut mae'r algorithm yn gweithio a beth yw'r fathemateg tu \^{o}l i'r algorithm. Yna fydd yna diwtorial o sut i'w defnyddio yn R ac yna tiwtorial arall yn Python.

Ar gyfer y wefan, mae dudalen cartref lle fydd yno linc i bob tiwtorialau yn y ddau Python ag R. Mae'r wefan yn cynnwys y tiwtorialau ag y wybodaeth angenrheidiol i allu gwneud yr algorithmau yn yr iaith rhaglennu o'ch dewis. Dyma yr wefan: \url{https://dysgupeirianyddol.github.io/}.

