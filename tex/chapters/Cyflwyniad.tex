\chapter{Cyflwyniad}\label{cha:introduction}

Mi fyddwn yn ysgrifennu fy mhrosiect blwyddyn ddiwethaf am dan algorithmau dysgu peirianyddol ag sut i'w defnyddio yn R ac Python. Fydd y prosiect yn cael ei ysgrifennu drwy gyfrwn Gymraeg. Mi fyddwn yn creu traethawd a gwefan i gyfathrebu gwybodaeth am yr algorithmau. Dwi am ysgrifennu am glystyru $k$-cymedr, atchweliad logistaidd ag ....

Bydd y tiwtorialau yn cael ei ddangos drwy'r ieithoedd rhaglennu Python ag R, gan taw nhw yw'r ddwy iaith fwyaf poblogaidd yn dysgu peirianyddol a gwyddor data yn \^{o}l arolwg Kaggle nol yn 2018 \cite{kagglesurvey}. Mae hefyd ganddynt trwyddedau am ddim, sy'n bwysig ar gyfer hygrychedd.

\section{Beth yw Dysgu Peirianyddol}\label{sec:intro_pd}

Yn syml, dysgu peirianyddol yw algorithm i optimeiddio rhyw feini prawf gan ddefnyddio data. Mae rhain yn cynnwys model wedi'i ddiffinio o rhai paramedrau mesuradwy, a'r darn dysgu fydd i optimeiddio gyda pharch tuag at y paramedrau hyn. Gall y model fod yn un disgrifiadol o'r data, neu un sy'n rhagfynegi rhyw agwedd o'r data. Mae dysgu peirianyddol yn defnyddio ystadegaeth i adeiladu'r modelau mathemategol, ac mae cyfrifiadureg yn edrych mwy i mewn i effeithiolrwydd y proses.\cite{dysgu-peirianyddol2}

Dydi dysgu peirianyddol ddim yn faes newydd, mae wedi bod o gwmpas ers y 50au pan wnaeth Arthur Samuel o IBM creu rhaglen ar y cyfrifiadur i chwarae'r gem `checkers'. Yr amser hwn cafodd y term ei bathu. Yna drwy ddatblygiadau technolegol diweddar, mae posibilrwyddau o ddysgu peirianyddol bron yn ddiddiwedd. Ers cael ei sefydlu yn y 50au, cymerodd tan 1997 i ddatblygu rhaglen a all guro'r chwaraewr gwyddbwyll orau yn y byd. Nid yn unig yw dysgu peirianyddol yn cael ei ddefnyddio i greu rhaglenni gemau, mae nawr yn cael ei ddefnyddio ym mhob math o raglenni. % megis...

Mae sylfaeni ddysgu peirianyddol yn cael eu defnyddio yn prosesu iaith naturiol. Mae'r proseses yma yn cael ei ddefnyddio i wneud ffwythiannau megis adnabod lleferydd, creu tecst i leferydd a chyfieithiad peirianyddol.
Yn ogystal mae dysgu peirianyddol yn cael ei ddefnyddio i brosesu lluniau, mae'r defnydd yma'n cael ei weld yn aml gyda systemau anabod wynebau.

Mae dysgu peirianyddol yn is-set o ddeallusrwydd artiffisial. Mae deallusrwydd artiffisial wedi tyfu yn esbonyddol yn ddiweddar gydag dysgu peirianyddol. Mae deallusrwydd artiffisial yn y newyddion drwy'r adeg oherwydd datblygiadau parhaus. Yn ddiweddar rydym wedi gweld moduron heb yrrwr, roboteg glyfar, a `chat bots'.
% pam is-set? beth yw'r wahaniaeth?

Categoreiddiwn algorithmau dysgu peirianyddol i ddwy fath - dysgu o dan orchwyliaeth, a dysgu heb orchwyliaeth. Maent yn wahanol yn eu pwrpas a'u dulliau.

\subsection{Dysgu dan Oruchwyliaeth}

Gadewch i'n data for nifer o barau o bwyntiau $(\mathbf{x}_{i},y_{i})$. Yn y fan hyn mae $y_{i} \in \alpha$ yn cael ei alw yn labeli, gall y labelau yma fod yn arwahanol neu ddi-dor. Fydd $\mathbf{x}_{i} \in  \beta$ yn fector gyda gwerthoedd ar gyfer priodweddau gwahanol. Y nod ar gyfer dysgu dan oruchwyliaeth yw dysgu'r mapiad o $\mathbf{x}$ i $y$ gan ddefnyddio'r set ymarfer.\cite{dysgu-peirianyddol}

Gwnawn hyn trwy hollti'r data i mewn i set ymarfer a set prawf, lle mae'n bwysig i dybio pob p\^{a}r wedi'i samplu'n annibynnol a'i dosbarthu o ddosraniad dros $\alpha \times \beta$. Defnyddiwn y set ymarfer i rhedeg yr algorithm dysgu peiranyddol a chanfod y mapiad. Defnyddiwn y set prawf er mwyn gwirio'r mapiad hyn. Fel arfer defnyddiwn tua 70\% o'r data fel y set ymarfer, a'r 30\% gweddill fel y set prawf, er taw tra-baramedr (hyper-parameter) yr hon ac felly gellid ei ddewis.

Mae'r ffurf wahanol o ddysgu dan oruchwyliaeth yn cael ei rhannu i ddau faes yn \^{o}l sut fath o ddata sydd gennym. Os yw'r label yn data arwahanol, mae gennym ddosbarthiad (classification). Os yw'r label yn data di-dor, mae gennym atchweliad (regression). Mae yna lwyth o wahanol fathau o ddosbarthiadau ag atchweliadau; dyma ambell o enghreifftiau ohonynt:

\begin{multicols}{2}
\textbf{Dosbarthiad:}

\begin{itemize}
	\item Coed penderfyniadau
	\item Dosbarthiad na\"{i}f Bayes 
	\item $K$ cymydog agosaf
\end{itemize} 

\textbf{Atchweliad:}

\begin{itemize}
	\item Atchweliad logistaidd
	\item Atchweliad llinol
	\item Atchweliad Poisson
\end{itemize}
\end{multicols}

\subsection{Dysgu heb Oruchwyliaeth}

Gadewch i'n data fod $X = (x_{1}, \dots, x_{n}) $ dynodi $n$ enghreifft lle mae $x_{i} \in \gamma$ ag $i \in \{ 1, \dots, n \}$. Tybiwn fod yr enghreifftiau $x_{i}$ wedi'i samplu'n annibynnol a'i dosbarthu o ddosraniad unfath ar $\gamma$. Y nod o ddysgu heb oruchwyliaeth yw amcangyfrif dwysedd o'r dosraniad ar $\gamma$.\cite{dysgu-peirianyddol}

Mae'r fatha boblogaidd o ddysgu heb oruchwyliaeth yn cymryd ffurf wannach o'r syniad hyn. Dyma ambell i enghraifft ar ffurfiau gwahanol o ddysgu heb oruchwyliaeth:

\begin{itemize}
	\item Clystyru
	\item Lleihad dimensiwn
	\item Model Markov cudd
\end{itemize}

% Beth am dysgu atgyfnerthol?

\subsection{Eraill}

Mae dysgu dan oruchwyliaeth rannol yn cymryd priodweddau o'r ddwy ffordd o ddysgu. Y ffurf fwyaf traddodiadol o hyn yw cael data wedi'i labelu a data heb ei labelu ac yno dilyn proses hunan ddysgu.

%Fuzzy

\section{Pam}

% Angen rhywbeth bach fan hyn

\subsection{Be sydd yna yn barod?}

\begin{center}
\begin{tabular}{ | c | c | c | c | c | }
\hline
Adnodd & Awdur & Fformat & Cynulleidfa Darged & Linc\\
\hline
Cwrs `Cyfrifiadureg ar gyfer Mathemateg' yn Python & Dr Vince Knight a Dr Geraint Palmer & Tiwtorial ar wefan & Dechreuwyr mewn codio Python & \cite{Cyfrifiadureg-maths}\\
\hline
Cyrsiau allgyrsiol yn Scratch, HTML, CSS ag Python & Code Club & Tiwtorialau ar y w\^{e} & Plant rhwng $9$ ac $13$ & \cite{codeclub} \\
\hline
Adnoddau ar lawer o testynau wahanol yn cyfrifiadureg & Technocamps & Mewn ffurdd pdf & Plant yn ysgol/coleg & \cite{technocamps} \\
\hline
Cwrs mewn sgiliau ymchwil cyfrifiadurol & Dr Geraint Palmer & Tiwtorialau ar y w\^{e} & Ymchwilwyr & \cite{python-sgiliauymchwil} \\
\hline
Gwybodath ar gydrannau pwysicaf technolegau iaith & Uned technoleg iaith Prifysgol Bangor a Cymen Cyf. & Llawlyfr ar y w\^{e}  & Myfyrwyr, datblygwyr neu academyddion heb gefndir yn y maes & \cite{technolegau-iaith} \\
\hline
ap Botio i hybu rhaglennu i plant & Tinopolis & ap ar cynnyrch apple & Plant rhwng $7$ ac $11$ & \cite{botio} \\
\hline
Enghreifftiau a thiwtorialau o cynnyrch technolegau iaith  &  Uned technoleg iaith Prifysgol Bangor a Cymen Cyf. & Ystorfa ar github & Myfyrwyr, datblygwyr neu academyddion heb gefndir yn y maes & \cite{github-technolegauiaith} \\
\hline
Tiwtorialau Python & Dr Geraint Palmer a Stephanie Jones & Fideos ar Youtube & Dechreuwyr yn rhaglennu gyda Python & \cite{youtube} \\
\hline
\hline
Nodiadau agored am algebra llinol & Dr Alun Morris & Nodiadau (pdf) & Myfyrwyr prifysgol & \cite{Algebra-llinol}\\
\hline
Amrhyw o adnoddau Mathemategol yn dilyn cwricwla CBAC & Dr Gareth Evans & Wefan gyda linciau i wahanol fathau o adnoddau & Myfyrwyr rhwng $11$ a $18$ & \cite{mathemateg} \\
\hline
\end{tabular}
\end{center}

Mae'r tabl uchod wedi'i rhannu i adnoddau codio Cymraeg ag i adnoddau Mathemategol Cymraeg. Fel gwelwn yn ebrwydd, mae'r nifer o adnoddau codio yn llawer fwy nag rheina o adnoddau mathemategol. Mae'r adnodd \cite{Cyfrifiadureg-maths} yn cychwyn da i unrhyw un sydd gyda diddordeb o gychwyn codio yn Python, mae'n defnydd gwych ar gyfer myfyrwyr Mathemateg israddedig gan ei fod yn benodol i Fathemateg. Yn ogystal i'r cwrs yma mae gennym hefyd bach o gyflwyniad i Python ar y wefan Sgiliau Ymchwil Ailhynhyrchiadwy \cite{python-sgiliauymchwil}, eb bod hyn yn canolbwyntio mwy ar sgiliau fwy eang megis rheolaeth fersiwn a datblygu meddalwedd ymchwil.

Gwelwn yn ogystal fod yna adnoddau ar gyfer rhaglennu yn Python yn Gymraeg ar gael ar YouTube. Ar sianel Geraint Palmer \cite{youtube} mae yna gasgliad o diwtorialau Python. Mae'r tiwtorialau yn cychwyn gyda'r sylfaen o raglennu yn Python ag yn gorffen gyda mynd dros yr algorithm genetig, sy'n mynd law yn llaw gyda fy mhrosiect.
Mae'r gan y wefan codeclub amrywiaeth eang o adnoddau ar gyfer codio. Mae yna gyrsiau ar HTML, CSS, Python a Scratch. Yn ogystal mae yna brosiectau pellach sy'n gweithredu Raspberry Pi. 

Gwelwn gydag adnodd technocamps \cite{technocamps}, fod yna amrywiaeth eang o adnoddau yn fan hyn fyd. Mae'r wefan yn wedi'i thargedu i oedran h\^{y}n na'r wefan codeclub. Mae gan technocamps cyrsiau cychwynnol fel y canlynol:

\begin{enumerate}
	\item CS 101
	\item Deallusrwydd Artiffisial
	\item Greenfoot (Java)
	\item Python
	\item Scratch
\end{enumerate} 

Ar wefan technolegau iaith \cite{technolegau-iaith}, mae yna gyflwyniad i'r darnau fwyaf sylfaenol i dechnolegau iaith. Mae yna lawlyfr yn cynnwys gwybodaeth am ddeallusrwydd artiffisial, dysgu dwfn a phrosesu iaith naturiol. Yn rhedeg yn gyfagos i'r wefan yma yw'r ystorfeydd ar GitHub \cite{github-technolegauiaith}. Yn yr ystorfeydd mae yna diwtorial ar sut i greu robot sgwrsio drwy ``turing test lessons''. % Esbonia?

Yn y siop ap Apple, fedrem lawrlwytho'r ap botio \cite{botio}. Mae'n ap ar gyfer plant sydd \^{a} diddordeb cael i mewn i rhaglennu.

Ar yr ochr mathemategol, mae'r adnodd am ddim gan Alun Morris \cite{Algebra-llinol} yn trafod y sylfeini eu hangen i astudio algbra llinol yn mhrifysgol. Mae'r adnodd yma yn enghraifft wych o'r fath o adnoddau mae'r iaith Gymraeg angen fwy ohono. Mae yna brinder iawn ar adnoddau o'r ansawdd yma i'r lefel yma o addysg.
Serch hynny mae yna gyfoeth o adnoddau mathemateg Cymraeg ar gyfer addysg o dan $18$ oed, gwelwn hyn gyda'r nifer mawr o adnoddau ar wefan Dr Gareth Evans o Ysgol Creuddyn \cite{mathemateg}.  

% Beth am yr adnodd dysgy peirianyddol cymrage??

Gwelwn fod yna llawer o adnoddau ar gyfer y maes mathemateg a chyfrifiadureg ar gyfer addysg ysgol. Mae yna ddigon o adnoddau i alluogi cenhedlaeth newydd i astudio'r sylfaen gofynnol drwy'r ysgol i astudio unrhyw un o'r ddau yn brifysgol. Unwaith fyddem yn cyrraedd lefel addysg prifysgol, mae'r prinder yn amlwg. Gan fod testunau dysgu peirianyddol yn cynnwys mathemateg o lefel uwch, mae rhaid cael sylfaen datblygedig o destunau mathemateg i'w ddeall. Gallwn ddweud yr un peth am yr angenrheidrwydd o sylfaen datblygedig o gyfrifiadureg. Y broblem sydd gennym yw does yna ddim yr adnoddau yw gwneud gyda mathemateg lefel uwch nag rhaglennu yn R yn y Cymraeg.


\subsection{Pam Cymraeg?}

I gychwyn, rwyf eisiau creu gwefan sy'n cynnwys tiwtorialau Cymraeg am algorithmau dysgu peirianyddol oherwydd y prinder ohonyn. Mae yna gymaint o adnoddau ar gyfer pob mathau o algorithmau drwy Saesneg ond yn Gymraeg, does yna ddim byd o'r fath! Fel gwelwn yn y tabl o adnoddau cynnar, does yna ddim llawer o adnoddau Cymraeg sy'n cynnwys deunydd datblygedig ar gael i'r cyhoedd. O fy mhrofiad personol i, rwyf yn gwybod fod prifysgolion yng Nghymru gydag adnoddau drwy'r cyfrwng Cymraeg ar gael i'w myfyrwyr nhw yn unig. Teimlaf fod hyn yn atal y parhad o addysg ar \^{o}l i fyfyrwyr gorffen eu hacadem\"{i}au ohwerydd nad oed fod adnoddau am ddim ar gael.  

Rheswm arall i wneud yn Gymraeg yw cefnogi prosiect y llywodraeth a'r wlad i gael miliwn o siaradwyr Cymraeg erbyn 2050 \cite{cymraeg2050}. I gymhorthi'r prosiect yma cafodd cynllun ei rhoi allan yn hybu'r defnydd a chynnyrch o adnoddau technolegol Cymraeg \cite{cymraeg2050tech}. Yn ogystal i greu adnoddau mae'r cynllun yn targedu y datblygiad o ddeallusrwydd artiffisial drwy edrych ar brosesu iaith naturiol yn bennaf. Mae creu'r adnodd yma yn Gymraeg yn bodloni'r ddwy adran o'r cynllun, gobeithio bydd yn hybu fwy i gychwyn creu adnoddau addysg uwch yn y maes yma. 

\section{Strwythyr}

Fydd y traethawd yma yn cynnwys tair pennod ar algorithm dysgu peirianyddol. Yn pob pennod fydd yna gefndir ar beth yw'r algorithm a pryd fyddem yn ei ddefnyddio, yn ogystal fydd yna gefndir ar sut mae'r algorithm yn gweithio a beth yw'r fathemateg tu \^{o}l i'r algorithm. Yna fydd yna diwtorial o sut i'w defnyddio yn R ac yna tiwtorial arall yn Python.

Ar gyfer y wefan, fydd yna dudalen cartref lle fydd yno linc i bob tiwtorialau yn y ddau Python ag R. Fydd y wefan yn cynnwys y tiwtorialau ag y wybodaeth angenrheidiol i allu gwneud yr algorithmau yn yr iaith rhaglennu o'ch dewis.
% ychwanegu linc i'r wefan.

