\chapter{Cyflwyniad}\label{cha:introduction}

Mi fyddwn yn ysgrifennu fy mhrosiect blwyddyn diwethaf am dan algorithmau dysgu peirianyddol ag sut i eu defnyddio yn R ac Python. Fydd y prosiect yn cael ei ysgrifennu drwy gyfrwn Gymraeg. Mi fyddwn yn creu traethawd a gwefan i dangos y wybodaeth am yr algorithmau. Dwi am ysgrifennu am glystyru $k$-cymedr, atchweliad logistaidd ag , fydd y rhain i gyd yn pennod eu hunain.    

\section{Beth yw Dysgu Peirianyddol}\label{sec:intro_pd}
%DIFFINIAD
%Effaith i'r byd go iawn
	%%NLP
	%%AI
	%%Image Processing
\subsection{Dysgu dan Oruchwyliaeth}


\subsection{Dysgu heb Oruchwyliaeth}


\subsection{Darllen Pellach}
%Semi-Supervised Learning
%Fuzzy

\section{Pam}

\subsection{Be sydd yna yn barod?}

\subsection{Pam Python ag R?}

\subsection{Pam Cymraeg?}


\section{Strwythyr}

Fydd y traethawd yma ymlaen yn cynnwys tair pennod ar algorithm dysgu peirianyddol. Yn pob pennod fydd yna gefndir ar beth yw'r algorithm a pryd fyddem yn ei ddefnyddio, yn ogystal fydd yna gefndir ar sut mae'r algorithm yn gweithio a beth yw'r Fathemateg t\^{y} \^{o}l i'r algorithm. Yna fydd yna diwtorial o sut yw defnyddio yn R ac yna tiwtorial arall yn Python.

Ar gyfer y wefan, fydd yna dudalen cartref lle fydd yno linc i bob tiwtorialau yn y ddau Python ag R. Fydd y wefan yn cynnwys y tiwtorialau ag y wybodaeth angenrheidiol i allu gwneud yr algorithmau yn yr iaith rhaglennu o eich dewis.

